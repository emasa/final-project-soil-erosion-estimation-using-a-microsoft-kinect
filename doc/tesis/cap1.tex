\chapter{Introducci\'{o}n}

\section{Motivaciones}
\label{S:motivaciones}

%%% redactar mejor el primer parrafo haciendo enfasis en porque los modelos son a escala reducida  
Los modelos f\'{i}sicos son una de las principales herramientas que cuenta la ingenier\'{i}a para la medici\'{o}n y estudio de la erosi\'{o}n, as\'{i} como la acumulaci\'{o}n y transporte de sedimentos.
Tradicionalmente, la medici\'{o}n de variables sedimentarias en laboratorio ha consistido en el relevamiento manual de puntos, generalmente distribuidos sobre una grilla equidistante, por medio de un nivel \'{o}ptico y una mira telesc\'{o}pica.
%%% Ver si redactamos mejor lo de los errores
Esta metodolog\'{i}a, de car\'{a}cter intrusiva, tiene errores intr\'{i}nsecos generados por la intervenci\'{o}n humana y restricciones relativas a los instrumentos de medici\'{o}n; algunas fuentes de error son la incorrecta verticalidad de la mira, el inexacto apoyo del palpador sobre el modelo, errores en las lecturas, problemas en la verticalidad del nivel debido a las condiciones del terreno, entre otros.
%%% Cambiar la expresion trade-off por una castellano, revisar lo de los perfiles transversales 
Ademas, la precisi\'{o}n de posteriores productos (como pueden ser las curvas de nivel, modelos en 3D, perfiles transversales) y los an\'{a}lisis que puedan ser realizados utilizando dichos derivados, se veran afectados por la densidad de puntos relevados y la elecci\'{o}n de los mismos. Esta limitaci\'{o}n produce un trade-off entre la densidad de puntos, el area relevada y el tiempo dedicado a cada ensayo, que el ingeniero debe afrontar a la hora de realizar un ensayo. \\

En el campo de la robotica, la construccion de mapas 3D del entorno ha sido considerado por a\~{n}os una tarea fundamental para su aplicacion en la navegacion autonoma y la manipulacion de objetos por parte de un robot. Diferentes sensores han sido utilizados para lograr este proposito, entre ellos se puede mencionar el uso de sonares[ref], escaneres laser[ref], camaras estereoscopicas[ref], camaras RGB-D[ref], etc. 

%%% Breve introduccion en la tecnica de SLAM

En los ultimos a\~{n}os, sensores RGB-D como el dispositivo Microsoft Kinect han aparecido en la industria del entretenimiento. Sus prestaciones tecnicas y costo asequible lo han convertido en una solucion idonea para ser utilizada en diferentes aplicaciones de Computer Vision, como es el caso de la generacion de mapas en 3D. \\
%%% agregar un poco mas de informacion sobre los dispositivos RGB-D

%%% en lugar de mapa en 3D habria que decir DEM (modelo digital de elevacion)
En este trabajo se propone una nueva tecnica para la medicion de erosion en modelos a escala reducida mediante la generacion automatica de un mapa en 3D utilizando camaras RGB-D de bajo costo y tecnicas de Computer Vision.
%%% agregar las ventajas de la t\'{e}cnica digital (enfocado en las ventajas : densidad de datos, velocidad de medici\'{o}n, automatizaci\'{o}n, mayor cobertura, no intrusiva)
 

\section{Objetivos}
\label{S:objetivos}

%%% en lugar de mapa en 3D habria que decir DEM (modelo digital de elevacion)
El principal objetivo de este trabajo es la implementacion de una sistema que permita generar un mapa en 3D del modelo fisico por medio de un sensor RGB-D, para ser utilizado en los ensayos de medicion de erosion en laboratorio. \\
%%% Agregar que debe permitir la visualizacion para poder realizar las capturas
El sistema debera capturar las nubes de puntos, generar el modelo 3D del area de trabajo, filtrar posibles inconsistencias y escalar los datos de modelo a prototipo. \\
Se validara la nueva t\'{e}cnica con respecto a la tecnica tradicional para realizar una evaluacion de su precision. \\
Se busca desarrollar una librer\'{i}a que permita crear futuras aplicaciones y sirva como base para crear mejores soluciones al problema de la medicion de erosion.

%%% Local Variables: 
%%% mode: latex
%%% TeX-master: "template"
%%% End: 
