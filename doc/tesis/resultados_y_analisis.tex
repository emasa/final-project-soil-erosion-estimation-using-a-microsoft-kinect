\chapter{Resultados y Analisis}

Se presentan resultados obtenidos en el ensayo REFERENCIAR[ENSAYO] y se utilizan para validar la tecnica propuesta.

En este ensayo se registro una area de 4.65 m**2 (equivalente en prototipo a 19700 m**2) aguas arriba del dique. Se obtuvo una densidad de puntos de ... 

\section{Comparacion tecnica tradicional y tecnica digital}

\subsection{Comparacion de perfiles}
Se comparan perfiles obtenidos con ambas tecnicas analizando la precision al relevar cotas de erosion maxima, y la precision global de todo el perfil.

En la figuras REFERENCIAR[PERFIL], se pueden observar que en algunas zonas, la tecnica tradicional mide erosion de forma imprecisa. Esto se debe principalmente a 2 razones : errores en el uso de los instrumentos de medicion, y una eleccion incorrecta de los puntos relevados (atraves de inspeccion visual de la superficie). Estas falencias tienen mayores consecuencias cuando afectan a la cota de maxima erosion.

A partir de las diferentas absolutas para los valores de elevacion entre ambas tecnicas, se obtuvo que : 
\begin{itemize}

\item Las maximas diferencias observadas fueron de XXX m en prototipo (YYY mm en modelo). 

\item Las diferecias minimas tienden a cero.

\item El promedio de las diferencias es aproximadamente ZZZ m en prototipo (WWW mm en modelo) \\

\end{itemize}

Esta comparacion pone en evidencia que la resolucion del modelo, obtenida con la tecnica digital, es superior en zonas donde los datos relevados con la tecnica tradicional no han sido cuidadosamente elegidos o el terreno es muy irregular; mientras que se obtienen resultados similares cuando no se incurre en estos errores.

\subsection{Comparacion de superficies}

\subsection{Analisis de correlacion}

\section{Control de estructuras del dique}

Por medio del ensayo REFERENCIAR[ENSAYO], se muestra que la tecnica propuesta en este trabajo captura con precision aceptable las estructuras del dique, lo que permite utilizarla como referencias visuales en el estudio de la erosion. En las figuras XXX y YYY, se muestran perfiles sobre el muro de encauzamiento, y se observa que se mantiene la horizontalidad propia de la escena y su cota prototipo a 1377.7 m s.n.m. Se observa que la mayor parte de la estructura se representa de forma continua. En algunas areas, con cambios abruptos de altitud (Ej: borde del muro), pueden producirse inconsistencias al ser relevadas desde puntos de vista muy diferentes.

