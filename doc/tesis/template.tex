%%%%%%%%%%%%%%%%%%%%%%%%%%%%%%%%%%%%%%%%%%%%%%%%%%%%%%%%%%%%%%%%%%%%%%%%%%%%%%%%
%\documentclass[12pt,papel,twoside]{tesis}
\documentclass[12pt,screen,twoside,pagebackref]{tesis}
%\documentclass[12pt,papel,singlespace,oneside]{tesis}
%\documentclass[12pt,papel,preprint,singlespace,oneside]{tesis}


%%%%%%%%%%%%%%%%%%%%% Paquetes extra %%%%%%%%%%%%%%%%%%%%%%%%%%%%%%%%%%%%%%%%%%%
% Por conveniencia: aqu\'{\i} puede cargar todos los paquetes y definir los comandos 
% que necesite
\usepackage{extra}
%%%%%%%%%%%%%%%%%%%%%%%%%%%%%%%%%%%%%%%%%%%%%%%%%%%%%%%%%%%%%%%%%%%%%%%%%%%%%%%%
%%%%%%%%%%%%%%%%%%%%% Informacion sobre la tesis %%%%%%%%%%%%%%%%%%%%%%%%%%%%%%%
\title{Software para medici\'{o}n de erosi\'{o}n y sedimentaci\'{o}n en modelos f\'{i}sicos por medio de relevamiento digital de superficies}
\author{Emanuel Sanchez Aimar}
\director{Mgs. Ing. Mariana Pagot}
\codirector{Dr. Oscar Bustos}
\carrera{Licenciatura en Ciencias de la Computaci\'{o}n}
\laboratorio{}
\grado{}
\palabrasclave{A definir}
\keywords{to define}
% Si queremos poner la fecha manualmente:
% \date{Diciembre de 2099}

%%%%%%%%%%%%%%%%%%%%%%%%%%%%%%%%%%%%%%%%%%%%%%%%%%%%%%%%%%%%%%%%%%%%%%%%%%%%%%%%
%\titlepagefalse % Si no quiere compilar la portada descomente esta linea
%\includeonly{apendices} % Compilar s\'{o}lo estos archivos 
\graphicspath{{figs/}} % Lugar donde encontrar las figuras generales (se puede poner uno en cada cap{\'{\i}}tulo)
%%%%%%%%%%%%%%%%%%%%%%%%%%%%%%%%%%%%%%%%%%%%%%%%%%%%%%%%%%%%%%%%%%%%%%%%%%%%%%%%


\begin{document}

% Dentro del environment 'preliminary' va:
% la dedicatoria, resumen, abstract, indices

\begin{preliminary}

% Escriba su dedicatoria
\dedicatoria{
A mi familia\\
A mis amigos\\
A todos los que me conocen\\
A toda esa otra gente que no
}

%%% \'{I}ndices %%%%

\begin{abreviaturas}
                                %Abreviaturas
\end{abreviaturas}

\tableofcontents                %\'{I}ndice

\listoffigures                  %Figuras

\listoftables                   %Tablas

\begin{resumen}

El presente trabajo plantea el desarrollo de una técnica experimental que permite medir la erosión y sedimentación en modelos físicos a escala reducida en laboratorio mediante la generación automática de un mapa 3D que representa la condición resultante de un ensayo hidráulico, utilizando una cámara RGB-D Microsoft Kinect. \\
Esta técnica se presenta ante la necesidad de realizar un relevamiento de erosión de un modo más preciso y eficiente respecto de la técnica utilizada tradicionalmente, que consiste en un relevamiento manual de puntos, generalmente haciendo uso de un nivel óptico y una mira milimétrica.\\ 
La técnica propuesta exhibe varios aspectos que superan al enfoque tradicional. En primer lugar, es una solución no intrusiva, es decir, no produce alteraciones sobre la condición del modelo. Además, mejora significativamente la resolución espacial del área medida, permite cubrir áreas extensas y disminuye el tiempo de medición de cada escenario ensayado. Los resultados obtenidos muestran que la solución propuesta permite realizar mediciones de erosión y sedimentación con una precisión superior a la técnica tradicional. \\

\end{resumen}

\end{preliminary}


% Podemos usar cualquiera de los dos comandos: \input o \include para incluir el texto
\chapter{Introducci\'{o}n}

\section{Motivaciones}
\label{S:motivaciones}

%%% redactar mejor el primer parrafo haciendo enfasis en porque los modelos son a escala reducida  
Los modelos f\'{i}sicos son una de las principales herramientas que cuenta la ingenier\'{i}a para la medici\'{o}n y estudio de la erosi\'{o}n, as\'{i} como la acumulaci\'{o}n y transporte de sedimentos.
Tradicionalmente, la medici\'{o}n de variables sedimentarias en laboratorio ha consistido en el relevamiento manual de puntos, generalmente distribuidos sobre una grilla equidistante, por medio de un nivel \'{o}ptico y una mira telesc\'{o}pica.
%%% Ver si redactamos mejor lo de los errores
Esta metodolog\'{i}a, de car\'{a}cter intrusiva, tiene errores intr\'{i}nsecos generados por la intervenci\'{o}n humana y restricciones relativas a los instrumentos de medici\'{o}n; algunas fuentes de error son la incorrecta verticalidad de la mira, el inexacto apoyo del palpador sobre el modelo, errores en las lecturas, problemas en la verticalidad del nivel debido a las condiciones del terreno, entre otros.
%%% Cambiar la expresion trade-off por una castellano, revisar lo de los perfiles transversales 
Ademas, la precisi\'{o}n de posteriores productos (como pueden ser las curvas de nivel, modelos en 3D, perfiles transversales) y los an\'{a}lisis que puedan ser realizados utilizando dichos derivados, se veran afectados por la densidad de puntos relevados y la elecci\'{o}n de los mismos. Esta limitaci\'{o}n produce un trade-off entre la densidad de puntos, el area relevada y el tiempo dedicado a cada ensayo, que el ingeniero debe afrontar a la hora de realizar un ensayo. \\

En el campo de la robotica, la construccion de mapas 3D del entorno ha sido considerado por a\~{n}os una tarea fundamental para su aplicacion en la navegacion autonoma y la manipulacion de objetos por parte de un robot. Diferentes sensores han sido utilizados para lograr este proposito, entre ellos se puede mencionar el uso de sonares[ref], escaneres laser[ref], camaras estereoscopicas[ref], camaras RGB-D[ref], etc. 

%%% Breve introduccion en la tecnica de SLAM

En los ultimos a\~{n}os, sensores RGB-D como el dispositivo Microsoft Kinect han aparecido en la industria del entretenimiento. Sus prestaciones tecnicas y costo asequible lo han convertido en una solucion idonea para ser utilizada en diferentes aplicaciones de Computer Vision, como es el caso de la generacion de mapas en 3D. \\
%%% agregar un poco mas de informacion sobre los dispositivos RGB-D

%%% en lugar de mapa en 3D habria que decir DEM (modelo digital de elevacion)
En este trabajo se propone una nueva tecnica para la medicion de erosion en modelos a escala reducida mediante la generacion automatica de un mapa en 3D utilizando camaras RGB-D de bajo costo y tecnicas de Computer Vision.
%%% agregar las ventajas de la t\'{e}cnica digital (enfocado en las ventajas : densidad de datos, velocidad de medici\'{o}n, automatizaci\'{o}n, mayor cobertura, no intrusiva)
 

\section{Objetivos}
\label{S:objetivos}

%%% en lugar de mapa en 3D habria que decir DEM (modelo digital de elevacion)
El principal objetivo de este trabajo es la implementacion de una sistema que permita generar un mapa en 3D del modelo fisico por medio de un sensor RGB-D, para ser utilizado en los ensayos de medicion de erosion en laboratorio. \\
%%% Agregar que debe permitir la visualizacion para poder realizar las capturas
El sistema debera capturar las nubes de puntos, generar el modelo 3D del area de trabajo, filtrar posibles inconsistencias y escalar los datos de modelo a prototipo. \\
Se validara la nueva t\'{e}cnica con respecto a la tecnica tradicional para realizar una evaluacion de su precision. \\
Se busca desarrollar una librer\'{i}a que permita crear futuras aplicaciones y sirva como base para crear mejores soluciones al problema de la medicion de erosion.

%%% Local Variables: 
%%% mode: latex
%%% TeX-master: "template"
%%% End: 

\chapter{Revision del estado del arte}
\section{Aplicaciones en Computer Vision}

En el campo de Computer Vision se han desarrollado varias tecnicas que permiten la construccion de mapas 3D utilizando difenrentes tipos de sensores, entre ellos, sensores laseres \cite{chou2013robotic}, camaras monoculares \cite{tomono2009robust}, camaras estereo, camaras RGB-D.

Desde la aparición del dispositivo Microsoft Kinect varios investigadores han desarrolado tecnicas para la construccion de mapas 3D aprovechando las caracteristicas.

\chapter{Capitulo sin nombre}


\chapter{El sensor Microsoft Kinect}
% redactar mejor
El sensor Microsoft Kinect, inicialmente dise\~{n}ado para la consola de juegos Microsoft Xbox 360 CITAR[MSKinect] fue lanzado en Noviembre de 2010.
Esta compuesto por una camara RGB, un sensor de profundidad, un array de microfonos y un mecanismo de inclinacion motorizado.
\\
AGREGAR IMAGEN

\section{Descripcion}
\label{S:descripcion-kinect}
La camara RGB produce un stream de datos de 24 bits por pixel, 8 bits por color. Su resolucion estandar es 640x480 pixels con una tasa de muestreo maxima de 30 FPS. \\
El sensor de profundidad esta compuesto por un emisor laser infrarrojo y un sensor CMOS monocromo. Produce un stream de datos de 11 bits. Posee una resolucion estandar de 640x480 pixels a una tasa de muestreo maxima de 30 FPS. Su rango util es de 0.4 a 4 metros. HAY DIFERENTES VERSIONES SOBRE LA CANTIDAD DE BITS, RANGO UTIL,  Y LA RESOLUCION MAXIMA, VER BIEN CUAL PONER \\
El campo de vision es de 57° horizontal y 43° vertical. \\
El resto de los sensores asi como el mecanismo de inclinacion no se exponen este trabajo debido a que no son relevantes para el funcionamiento de la aplicacion.

\section{Funcionamiento interno}
\label{S:funcionamiento-kinect}

El funcionamiento de la camara Kinect esta basado en la tecnologia de la empresa israeli PrimeSense CITAR[PrimeSense]. \\
Para obtener una imagen de profundidad el emisor laser emite un patron de puntos que es capturado por la camara infrarroja (sensor CMOS monocromo). AGREGAR FIGURA. \\
El proceso que se describe en la patente CITAR[PrimeSensePatent] esta formado por las siguientes etapas :
\begin{enumerate}
\item Capturar el patron de puntos para un conjunto de imagenes de referencia a diferentes distancias del plano del sensor.
\item Capturar el patron de puntos sobre una imagen de test de la region de interes.
\item Encontrar la imagen de referencia que mayor similitud tiene con la imagen de test utilizando Cross Correlation CITAR[CrossCorrelationWIKI].
\item Estimar el mapa 3D de la escena por medio de un proceso de triangulacion utilizando los desplazamientos entre la imagen de test y la imagen de referencia elegida. 
\end{enumerate}

En el caso del Microsoft Kinect, las imagenes de referencia han sido capturadas contra una superficie plana a distancias predifinidas y estan almacenadas en el dispositivo. El sensor devuelve la imagen de profundidad en forma de valores de disparidad que luego son traducidos a distancias en metros [VER: o milimitros], en un procedimento externo, mediante la conversion : AGREGAR FORMULA.

AGREGAR QUE LA IMAGEN DE PROFUNDIDAD ESTA SINCRONIZADA CON LA IMAGEN DE RGB QUE FUE PROCESADA CON UN FILTRO DE BAYER.

%\appendix
% \include{apend1}

\begin{biblio}
\bibliography{mibib}
\end{biblio}

\begin{postliminary}

\begin{seccion}{Publicaciones asociadas}
  \begin{enumerate}
  \item Simposio \textbf{ABC}, 2013
  \item Congreso \textbf{ABC}, 2013
  \item Revista  \textbf{ABC}, 2014 
  \end{enumerate}
\end{seccion}

%\begin{seccion}{Agradecimientos}
%A todos los que se lo merecen, por merecerlo
%\end{seccion}

\end{postliminary}

\end{document}

