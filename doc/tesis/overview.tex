\chapter{Visión general}

Desde una perspectiva global, para generar un mapa 3D de un modelo físico se considera necesario realizar las siguientes etapas :
\begin{itemize}

\item Captura de la imagen a partir de la cámara RGB-D.

\item Registración de la imagen. En esta etapa para unificar los puntos en un mismo sistema de coordenadas.

\item Eliminación de inconsistencias. La cámara puede relevar puntos sin información de profundidad o con mediciones incorrectas que deben ser filtrados para obtener mapas 3D consistentes. 

\item Conversión del sistema de coordenadas de modelo a prototipo. Inicialmente, el conjunto de imágenes (habiendo sido alineadas o no) están en un sistema de coordenadas que tiene su origen en el sensor RGB-D. En esta instancia, se cambia el sistema de coordenadas y se aplica un cambio de escala, con el fin de obtener una representación del mapa digital de elevaciones (MDE) del prototipo.

\end{itemize}

En este trabajo, la etapa de captura y registración se ejecutan incrementalmente hasta obtener un mapa 3D de la escena. Una vez finalizada la alineación, se realiza la eliminación de inconsistencias y finalmente la conversión a prototipo. En la figura REFERENCIAR[FIGURA] se ilustra el procedimiento implementado.

FALTA GRAFICO Y MAS CONTENIDO.

