\chapter{Vision general}

Desde una perspectiva global, para generar un mapa 3D de un modelo fisico se considera necesario realizar las siguientes etapas :
\begin{itemize}

\item Captura de la imagen a partir de la camara RGB-D.

\item Registracion de la imagen. En esta etapa para unificar los puntos en un mismo sistema de coordenadas.

\item Eliminacion de inconsistencias. La camara puede relevar puntos sin informacion de profundidad o con mediciones incorrectas que deben ser filtrados para obtener mapas 3D consistentes. 

\item Conversion del sistema de coordenadas de modelo a prototipo. Inicialmente, el conjunto de imagenes (habiendo sido alineadas o no) estan en un sistema de coordenadas que tiene su origen en el sensor RGB-D. En esta instancia, se cambia el sistema de coordenadas y se aplica un cambio de escala, con el fin de obtener una representacion del mapa digital de elevaciones (MDE) del prototipo.

\end{itemize}

En este trabajo, la etapa de captura y registracion se ejecutan incrementalmente hasta obtener un mapa 3D de la escena. Una vez finalazada la alineacion, se realiza la eliminacion de inconsistencias y finalmente la conversion a prototipo. En la figura REFERENCIAR[FIGURA] se ilustra el procediento implementado.

