\chapter{Conclusiones}

En este trabajo se ha desarrollado un sistema que permite medir la erosión y sedimentación en modelos físicos a escala reducida en laboratorio mediante la generación automática de un mapa 3D que representa la condición resultante de un ensayo hidráulico, utilizando una cámara RGB-D Microsoft Kinect \cite{microsoft-kinect}. \\
Es importante destacar que la técnica digital propuesta e implementada en el presente trabajo exhibe una serie de ventajas que la muestran superadora frente a la técnica tradicional. Hay dos aspectos que se mejoran utilizando esta técnica, el que se refiere a la interacción con el medio y el referido a las facilidades tecnológicas que esta solución aporta. El primero de ellos se debe a la capacidad que presenta la propuesta para realizar mediciones no intrusivas, es decir, que no altera la condición resultante del ensayo hidráulico. Por otro lado, el sistema propuesto presenta beneficios en la adquisición de datos porque mejora significativamente la resolución espacial del área relevada, a la vez que permite registrar áreas extensas y disminuir el tiempo de medición de los resultados para cada escenario ensayado. \\
Como se pudo evaluar en diferentes ensayos, se observa que la precisión de la técnica digital está ligada a una serie de factores, tanto externos como específicos al software empleado. Entre los factores externos que influyen sobre el sensor Kinect se distinguen la luminosidad de la escena, la nivelación de la guía-soporte de la cámara, la presencia de agua en la superficie, entre otros. Por otra parte, es necesario una correcta configuración de los algoritmos empleados así como la presencia de solapamiento entre las imágenes capturadas. No obstante, varias instancias de experimentación mostraron que las mediciones de erosión relevadas con la técnica digital presentan diferencias de 7,5 mm promedio (aproximadamente) respecto a la técnica tradicional, posicionando a la propuesta introducida en este trabajo, como una alternativa viable para realizar mediciones de erosión en laboratorio. \\
Cabe destacar que las técnicas de procesamiento de imágenes empleadas para implementar el procedimiento de registración y generación de un mapa 3D  resultaron satisfactorias para resolver el problema de la medición de erosión, y además, brindan la oportunidad de mejorar la solución propuesta en futuros trabajos. \\
Se concluye que la técnica digital, presentada en este trabajo, es una propuesta viable, y en muchas aspectos superadora, a la técnica tradicional para la medición de erosión y sedimentación en laboratorio, brindando novedosas posibilidades que podrán ser explotadas en futuros ensayos en el Laboratorio de Hidráulica (FCEFyN, UNC), asi como tambien, en otros laboratorios que consideren de interés este trabajo.