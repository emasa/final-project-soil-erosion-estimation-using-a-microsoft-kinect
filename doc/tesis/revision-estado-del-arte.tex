\chapter{Revision del estado del arte}

En los campo de Robotica y Computer Vision, la construccion de mapas 3D del entorno 
, ha sido por años, tema de investigacion. Diferentes sensores han sido utilizados con este objetivo, entre ellos, sensores laseres \cite{chou2013robotic}\cite{Montemerlo02fastslam}, camaras monoculares \cite{tomono2009robust}\cite{clemente_etal_rss2007}, camaras estereo \cite{Mei11}\cite{Konolige08}, entre otros.

Desde la aparicion de la camara RGB-D Microsoft Kinect (y posteriormente el Asus Xtion), ha crecido el interes en desarrollar tecnicas que aprovechan en conjunto las caracteristicas visuales y la informacion de profundidad que este dispositivo provee. 

Varios trabajos estan basados en un enfoque comun : alineacion espacial de frames consecutivos, deteccion de zonas que ya hayan sido visitadas y una alineacion global de todas los frames capturados. En particular, varios trabajos que utilizan informacion visual y de profundidad, se basan en la extraccion de caracteristicas visuales. Estas caracteristicas son emparejadas entre frames continuos para realizar su alineacion.
Tambien se  

etapas emparejamiento de caracteristicas comunes para realizar una alineacion espacial de frames consecutivos, deteccion de zonas que ya hayan sido visitadas y una alineacion global de todas los frames capturados. Para 

\cite{engelhardreal}