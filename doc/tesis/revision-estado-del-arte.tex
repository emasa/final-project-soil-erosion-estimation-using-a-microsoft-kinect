\chapter{Revisión del estado del arte}
\label{cap:estado-del-arte}

En las áreas de robótica móvil y visión por computador, la construcción de mapas 3D del entorno ha sido tema de investigación por años. Diferentes sensores han sido utilizados con este objetivo, entre ellos, sensores laseres\cite{chou2013robotic}\cite{Montemerlo02fastslam}, camaras monoculares\cite{tomono2009robust}\cite{clemente_etal_rss2007}, camaras estereo \cite{Mei11}\cite{Konolige08}, entre otros. La mayoría de los sistemas de construcción de mapas requieren de alineación espacial entre imágenes o \textit{frames} consecutivos, detección de zonas visitadas anteriormente y alineación global de todas las imágenes para construir mapas consistentes. \\

Desde la aparición del dispositivo Microsoft Kinect (y posteriormente el Asus Xtion Pro Live\cite{asus-xtion-pro-live}) se han presentado varias soluciones al problema de construcción de mapas 3D que aprovechan tanto las características visuales como la información de profundidad que este sensor provee. En Henry et al. (2010)\cite{henry2010rgb} siguen un enfoque de extracción y emparejamiento de características visuales distintivas entre pares de frames. La información 3D asociada a estas características permite estimar la traslacion y rotacion de la cámara en un procedimiento de aproximación y refinamiento de la \textit{pose} (posicion y orientacion de la cámara) con técnicas como ICP (\textit{Iterative Closest Point})\cite{Besl92}. En varios trabajos consultados se ha empleado este enfoque aportando modificaciones que logran mejorar su precisión y velocidad \cite{engelhard2011real}\cite{hogmanbuilding}\cite{fioraio2011realtime}\cite{6614623}. En este sentido, se cita el trabajo de Engelhard et al. (2011)\cite{engelhard2011real} en el cual se presenta un escaneo 3D de objetos y proponiendo utilizar SURF (\textit{Speed Up Robust Feateures}) \cite{bay2008speeded} a cambio de SIFT (\textit{Scale Invariant Features Transform}), empleado originalmente en \cite{henry2010rgb}, para la detección de características visuales. En Fioraio y Konolige (2011)\cite{fioraio2011realtime} plantean utilizar FAST (\textit{Features from Accelerated Segment Test})\cite{Rosten06machinelearning} y BRIEF (\textit{Binary Robust Independent Elementary Features})\cite{Calonder12} para detectar características, y permitir la construcción de mapas 3D en tiempo real. En estas propuestas, se implementa una técnica de optimización global, conocida como SLAM (\textit{Simultaneous Localization And Mapping}), que utiliza el enfoque anterior, para detectar zonas del entorno que ya han sido visitadas, lo que permite construir mapas globalmente consistentes. \\

Por otro lado, en el centro de investigación UC Davis KeckCaves (2012)\cite{arsandbox} utilizan una cámara Kinect y un proyector digital, ambos suspendidos sobre un cajón de arena y enfocando la superficie, para crear una herramienta de realidad aumentada. La Kinect releva la superficie del cajón de arena y se proyecta un mapa digital de elevaciones (DEM, por sus siglas en inglés) en diferentes colores en tiempo real. Esta herramienta permite crear modelos topográficos, visualizar las modificaciones producidas e incluso montar simulaciones de flujo que interactúan con la información topográfica. \\

El presente trabajo se inspira en la idea propuesta en \cite{arsandbox} para medir la erosión y sedimentación de una superficie utilizando una cámara Kinect montada sobre un modelo físico hidráulico fluvial construido en el Laboratorio de Hidráulica de la facultad de Ciencias Exactas Físicas y Naturales de la UNC. Para poder relevar una superficie extensa, se aplica la solución presentada en \cite{henry2010rgb} que ha probado ser una alternativa viable para la construcción de mapas 3D. Además, se introduce SURF y ORB (Oriented FAST and Rotated BRIEF)\cite{RubleeRKB11}, propuestos por \cite{engelhardreal} y \cite{fioraio2011realtime} respectivamente, para la detección de características visuales.
