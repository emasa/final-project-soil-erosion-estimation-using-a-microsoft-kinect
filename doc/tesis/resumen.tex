\begin{resumen}

El presente trabajo plantea el desarrollo de una técnica experimental que permite medir la erosión y sedimentación en modelos físicos a escala reducida en laboratorio, mediante la generación automática de un mapa 3D que representa la condición resultante de un ensayo hidráulico, utilizando una cámara RGB-D Microsoft Kinect. \\
Esta técnica se presenta ante la necesidad de realizar un relevamiento de erosión de un modo más preciso y eficiente respecto de la técnica utilizada tradicionalmente, que consiste en un relevamiento manual de puntos, generalmente haciendo uso de un nivel óptico y una mira milimétrica.\\ 
La técnica propuesta exhibe varios aspectos que superan al enfoque tradicional. En primer lugar, es una solución no intrusiva, es decir, no produce alteraciones sobre la condición del modelo. Además, mejora significativamente la resolución espacial del área medida, permite cubrir áreas extensas y disminuye el tiempo de medición de cada escenario ensayado. Los resultados obtenidos muestran que la solución propuesta permite realizar mediciones de erosión y sedimentación con una precisión superior a la técnica tradicional. \\

\end{resumen}