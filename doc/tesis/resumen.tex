\begin{resumen}
El presente trabajo plantea el desarrollo de una técnica experimental que permite medir la erosión y sedimentación en modelos físicos a escala reducida en laboratorio mediante la generación automática de un mapa 3D que representa la condición resultante de un ensayo hidráulico, utilizando una cámara RGB-D Microsoft Kinect \cite{microsoft-kinect}. Esta técnica se presenta ante la necesidad de realizar un relevamiento de erosión de un modo más preciso y eficiente respecto de la técnica utilizada tradicionalmente, que consiste en un relevamiento manual de puntos, generalmente haciendo uso un nivel óptico y una mira milimétrica. \\
La técnica propuesta se presenta como método alternativo superando a la tradicional debido a que, por un lado propone un mecanismo que no modifica la condición del modelo (es una técnica no intrusiva), y además mejora significativamente la resolución espacial del área medida, permite relevar áreas extensas y disminuye el tiempo de medición de los resultados para cada escenario ensayado. \\
Los resultados obtenidos muestran que la solución propuesta permite realizar mediciones de erosión y sedimentación con una precisión superior a la técnica tradicional, y permite obtener una representación de la condición de erosión con una resolución espacial sin precedentes. 
\end{resumen}

% \begin{abstract}%
% \end{abstract}

%%% Local Variables: 
%%% mode: latex
%%% TeX-master: "template"
%%% End: 
